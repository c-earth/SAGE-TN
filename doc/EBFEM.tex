\documentclass[preprint, 12pt]{revtex4-2}
\usepackage{amssymb}
\usepackage{amsmath}
\usepackage{amsfonts}
\DeclareMathOperator{\Tr}{Tr}
\usepackage{physics}
\usepackage{xcolor}
\usepackage{mathtools}

\def\thesection{\arabic{section}}
\def\thesubsection{\arabic{section}.\arabic{subsection}}
\def\thesubsubsection{\arabic{section}.\arabic{subsection}.\arabic{subsubsection}}
\numberwithin{equation}{section}

\begin{document}

\title{Entangled Basis Finite Element Method PDE solver\\for Quantum Computer}

\author{Abhijatmedhi Chotrattanapituk}
\affiliation{Quantum Measurement Group, MIT, Cambridge, MA, USA \\
            Department of Electrical Engineering and Computer Science, MIT, Cambridge, MA, USA}

\date{\today}

\maketitle
\section{Finite Element Method (FEM)}
\section{Tensor Network (TN)}
\section{Target Equation}
In this work, we will focus on a class of analytic partial differential equation (A-PDE) of a function of time and space, $u(t, q_i|i\in\{1, 2, \cdots, D\})$, where we use  with $\mathbf{q}$ be spatial vector, that is linear in time (LTA-PDE). This PDE can be written in the form
\begin{equation}
    D_tu + h\left(t, \mathbf{q}, u, \mathbf{\partial_q}u, \mathbf{D_q}^2u, \cdots \right) = 0,
\end{equation}
where $D_t$ is time derivative, $D_q$ is space derivative, and $h$ is an analytic function which can be written as
\begin{equation}
    h\left(t, q, u, D_qu, D_q^2u, \cdots\right) = \sum_{p_t, p_q, p_0,\cdots}h_{p_t, p_q, p_0,\cdots}t^{p_t}q^{p_q}u^{p_0}\left(D_qu\right)^{p_1}\left(D_q^2u\right)^{p_2}\cdots.
\end{equation}
Notice that $h_{p_t, p_q, p_0,\cdots}$ is a rank $\prod_{r=0} (r\cdot p_r)$ tensors.
\section{FEM Representation of LTA-PDE}
\section{Tensor Optimization}
\section{Matrix Product State (MPS) Representation}
\section{Implementation}

\end{document}