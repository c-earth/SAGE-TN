\documentclass[preprint, 12pt]{revtex4-2}
\usepackage{amsmath}
\usepackage{amsfonts}
\DeclareMathOperator{\Tr}{Tr}
\usepackage{physics}
\usepackage{xcolor}
\usepackage{mathtools}
\usepackage{amssymb}

\def\thesection{\arabic{section}}
\def\thesubsection{\arabic{section}.\arabic{subsection}}
\def\thesubsubsection{\arabic{section}.\arabic{subsection}.\arabic{subsubsection}}
\numberwithin{equation}{section}

\begin{document}

\title{Entangled Basis Finite Element Method PDE solver\\for Quantum Computer}

\author{Abhijatmedhi Chotrattanapituk}
\affiliation{Quantum Measurement Group, MIT, Cambridge, MA, USA \\
            Department of Electrical Engineering and Computer Science, MIT, Cambridge, MA, USA}

\date{\today}

\maketitle
\section{Finite Element Method (FEM)}

\newpage

\section{Tensor Network (TN)}

\newpage

\section{Target Equation}
In this work, we are considering a class of partial differential equation (PDE) that is linear in time (LT-PDE) as shown in Eq. of a tensor-valued function, $\mathbf{u}(t, \mathbf{q})$, of time, $t\in \mathbb{R}$, and $D$-dimensional space, $\mathbf{q}\in\mathbb{R}^D$, where we use boldface variables to indicate tensors.
\begin{equation}
    \mathbf{h}(t, \textbf{q}, \textbf{u}, ) + \sum_{r_t=1}^{R_t}\textbf{g}_{r_t}(t, \mathbf{q})\cdot\partial_t^{r_t}\mathbf{u} = 0
\end{equation}

\newpage

\section{FEM Representation of LTA-PDE}

\newpage

\section{Tensor Optimization}

\newpage

\section{Matrix Product State (MPS) Representation}

\newpage

\section{Implementation}

\end{document}